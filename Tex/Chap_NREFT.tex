\chapter{非相对论性有效场论(NREFT)简介\label{chap_NREFT}}
量子场论诞生的动机来源于相对论性量子力学出现的种种问题。作为固定粒子数的理论,相对论性量子力学把量子力学和狭义相对论相结合的尝试存在着不可避免的缺陷:当能量高于粒子质量时,理论无法描述反粒子的产生和湮灭,这也是Klein佯谬的来源。量子场论成功地解决了相对论性量子力学的问题,自洽的将量子力学和相对论结合了起来。于是,通常我们提及量子场论的时候,我们假设了洛伦兹对称性(Lorentz symmetry),将量子场论称作“量子力学+相对论”。但这并不是量子场论的全部。尽管相对论性量子场论(relativistic QFT)拥有各种复杂而优美的性质,量子场论本身并没有任何要求阻止我们构造一个非相对论性的量子场论。

我们可以对一个相对论性量子场论做非相对论近似来构造其对应的NREFT。
我们同时注意到,所有非相对论展开的参数都是速度。这意味着所有NREFT的数幂原则都是按速度展开的。
如果我们只保留最低阶的非相对论展开,不加入任何相对论修正,也就是说,如果我们假设伽利略对称性(Galilean symmetry)就是最基本的时空对称性,这么一个量子场论完全是可重整的,理论中不会出现发散\footnote{正如非相对论性量子力学中也会出现发散一样,拉氏量中不出现抵消项不代表所有的计算都不出现发散,例如高量纲算符的格林函数很可能出现发散。}\footnote{在某些场合中,这种有效场论被称作薛定谔有效场论(Schr\"odinger EFT)。}。当我们加入相对论修正后,我们得到了一个不可重整理论。但正如第一章中我们对有效场论之原则的阐述,我们可以按照数幂原则逐阶重整化消除发散。此外,尽管我们已经通过非相对论展开破坏了洛伦兹对称性,我们还可以通过重参数化(reparametrization)恢复洛伦兹对称性。这告诉我们,哪怕一个低能有效理论已经显式破缺掉了完整理论的对称性,完整理论的对称性仍然隐含在低能理论中。我们要注意到,非常类似于非相对论性量子力学,NREFT也是固定粒子数的。在NREFT中,粒子和反粒子是退耦的,二者之间没有相互作用,这意味着粒子和反粒子之间不能相互转化,而媒介粒子如光子、胶子也不能产生正负粒子对。这一点完全由NREFT的有效能标范围限制,由于其能标一定小于粒子质量,其能量不足以产生反粒子。以上的性质导致NREFT和相对论性量子场论在实际计算时有很大区别。


在本章中,我们将对NREFT(NRQED、NRQCD)的框架做简要的介绍。利用NREFT进行计算的细节以及NRQCD因子化将留至第三章和第四章作详细阐述。在\ref{Lg}节中,我们将给出NREFT的拉氏量及其性质,我们同时将给出推导NREFT拉氏量的方法;在\ref{HQET}节中,我们将推导HQET的拉氏量,我们同时将对比NRQED、NRQCD和HQET,比较三者的异同以及不同的应用场景。

\section{NREFT的拉格朗日量\label{Lg}}
\subsection{NREFT中的标度}


\subsection{NREFT的拉格朗日量和数幂原则}

\section{重夸克有效理论(HQET)\label{HQET}}