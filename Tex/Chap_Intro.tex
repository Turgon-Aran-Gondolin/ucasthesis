% LTeX: language=zh-CN
\chapter{引言}\label{chap:introduction}


我们生活的自然界中存在着各种不同的尺度,大至星球、星系,小至原子、亚原子粒子,每一个尺度都存在着丰富的物理现象。幸运的是,我们并不需要完全理解每一个尺度的物理才能对一个特定尺度的物理作出预言,每一个尺度的物理现象都可以只通过该尺度下的自由度描述。这种想法事实上非常符合直觉:我们在日常生活中用到经典力学,并不会考虑每一个分子、原子如何运动,后者的效应相对前者是非常微小的。
% 当我们考虑如何搬砖的时候,我们只会以砖头为基本自由度,而不会去考虑砖头内部的分子、原子怎么运动。尽管其它尺度的物理也有贡献,相比之下是微不足道的。
这种利用某一个尺度下最合适的自由度来研究最重要的物理的理论,我们通常称之为“有效理论”\cite{Georgi1993}。尽管没有被明确指出,在物理学的发展史中,物理学家们反复应用了这种有效理论的理念。举例来说,至今牛顿力学仍然是一门单独的学科,我们仍然在大量应用牛顿力学而不是相对论处理宏观低速的对象;尽管在上世纪初我们就已经得到了原子的内部结构,化学家们仍然多数以原子、分子而非电子、原子核为基本自由度进行研究。我们当然可以利用更基本的理论来进行计算,但是这样会带来很多不必要的计算量,同时也会使得低能标与高能标的物理混在一起,丢失了重要的细节。比如,当我们计算氢原子的兰姆位移(Lamb shift)时,我们可以用相对论性的量子电动力学(quantum electrodynamics,QED)来研究\cite{Weinberg2005},也可以用非相对论性量子电动力学(nonrelativistic quantum electrodynamics,NRQED)来研究\cite{Pineda1998},两种方法得到的结果完全一样;但是后者的计算量远远小于前者,同时,在NRQED中反粒子被完全退耦,我们可以清楚地看到兰姆位移的来源。在如今的物理学大厦中,有许多低能理论可以被称作更完整理论的有效理论,牛顿力学就是其中的典型代表。% TODO


这种有效理论的思想,应用到我们赖之以描述自然界的框架量子场论(quantum field theory,QFT)上,便是有效场论(effective field theory,EFT)。有效场论的发展始于上世纪六十年代Steven Weinberg对强子散射过程的探索\cite{Weinberg2009,Georgi1993}。在六十年代,量子色动力学(quantum chromodynamics,QCD)尚不存在,处理强子相互作用的主要手段是流代数(current algebra)。但是由于流代数的计算过于复杂,Weinberg基于手征对称性(chiral symmetry)写下了以$\pi$介子(pion)为基本自由度的手征拉氏量(chiral Lagrangian)\cite{Weinberg1967,Weinberg1968}。在此后几年中,Schwinger、Dashen、Weinstein等人也提出了类似的利用手征对称性破缺给出的有效拉氏量。在这个阶段,尽管手征拉氏量是基于QFT框架的,但是它仅仅是作为在微扰论最低阶重现流代数的一种简化计算的方法,仍不是有效场论。由于手征拉氏量是不可重整(nonrenormalizable)理论,在当时更高阶的计算是不可能的。上世纪70年代初Ken Wilson关于重整化群(renormalization group)和临界现象(critical phenomena)的思考\cite{Wilson1971a,Wilson1971b,Wilson1972,Wilson1975}对整个量子场论的发展造成了巨大的冲击,有效场论也不例外。Wilson提出了一种新的观点:他引入了一个显式的紫外截断,积掉了短程的自由度和大动量模,于是得到了描述长程物理的一个量子场论。这样的量子场论的拉氏量有显式的截断依赖,但是由其得到的物理量是不依赖于截断的。为了保持物理量不依赖于截断,引入所有可能的相互作用就是必须的,哪怕更基本的理论是可重整的,积掉短程物理之后也可能变得不可重整。这意味着和此前的认识相悖,可重整性并不是量子场论的必要条件。Weinberg进一步地意识到哪怕没有截断,只要拉氏量中包含了所有对称性允许的项,就一定会有足够的抵消项来消除出现的紫外发散。这提供了重整化不可重整理论的方法。由此出发,Weinberg写出了不再限于树图阶的手征有效理论\cite{Weinberg1979}。

在此后的几十年间,有效场论蓬勃发展,在粒子物理和强子物理领域,除了上面提到的手征有效场论(chiral EFT)及其扩展如重强子手征微扰论(heavy hadron chiral perturbation theory)等之外,诸如重夸克有效理论(heavy quark effective theory,HQET)、非相对论性量子色动力学(nonrelativistic quantum chromodynamics,NRQCD)、软共线有效理论(soft collinear effective theory,SCET)、四费米子相互作用、不稳定粒子有效场论(unstable particle EFT)、核子-核子有效场论(NNEFT)等基于粒子物理标准模型(standard model)的低能有效场论也对唯象研究产生了巨大影响。新物理研究同样依赖有效场论,例如将标准模型视为有效场论最低阶的标准模型有效场论(standard model EFT,SMEFT)。除了粒子物理外,有效场论也在核物理、原子分子物理、凝聚态物理、宇宙学、天体物理、流体力学等领域获得了广泛应用。

有效场论的观点对我们关于量子场论本身的认知也产生了巨大的冲击。我们认识到量子场论本身就是一种有效场论,当我们将路径积分中的泛函测度取连续极限的时候,我们实际上积掉了短程的信息,于是理论本身自带一个截断。但当我们做圈积分的时候,我们先将这个截断趋于无穷再对整个动量空间进行积分,这与路径积分给出的顺序恰恰相反。问题在于将截断取极限和积分是不可对易的,这导致了实际的圈积分出现了发散。我们进行重整化的过程恰恰是将有效场论的参数和完整理论/实验数据进行匹配的过程。

有效场论的理论基础可以由Weinberg给出的一个俗定理(folk theorem)表述:在微扰论的任意阶,对于一个给定的渐进态的集合,一个最一般的可能的拉氏量包含所有假设的对称性允许的项,这么一个拉氏量会给出一个最一般的满足解析性、微扰幺正性、集团分解原理和假设的对称性的S矩阵元(S-matrix elements)。尽管没有证明,目前还没有出现过这个定理的反例。只从这个定理出发,情况看似很绝望:我们必须要在拉氏量中引入所有对称性允许的算符来保证一个没有发散的S矩阵元,对于一个不可重整理论,我们就需要引入无穷多个算符,这是完全不现实的。但是我们还必须引入有效场论的另一个要素:数幂原则(power counting)。数幂原则告诉我们一个有效场论应该按照什么参数展开,相应的,拉氏量中的每一项对应这个参数的哪一阶。当我们给定一个数幂原则之后,对于我们想要达到的某一阶的精度,在拉氏量中我们只需要考虑能给出相同精度的项。这样,我们就只需要处理有限个数的算符来保证结果达到我们想要的精度。另一个要素在于拉氏量中各个算符前的耦合常数。哪怕我们可以根据对称性和数幂原则写出我们想要的拉氏量,拉氏量中的参数仍然是未知的。我们需要计算几个物理量,并利用实验数据或是完整理论\footnote{这里的完整理论的含义如下:我们可以从某一个理论出发,积掉部分短程的物理,得到一个所谓的自上而下(top-down)的有效场论,这种情况下我们最开始出发的理论一般称为完整理论。}来确定这些参数,这个过程通常称为匹配(matching)。同时,一个有效场论一定有其适用范围。这点可以由数幂原则看出:数幂原则的参数实际上就是有效场论的展开参数,这种展开一定只是在该参数的一定取值范围内成立。当展开参数过大时,有效场论的展开就不再成立,这种情况通常对应能标过高,于是有效场论失效\footnote{在某些情况下,展开参数并不对应能标,比如将Einstein-Hilbert作用量关于$h^{\mu\nu}$展开,这时候,展开参数过大意味着时空过于弯曲。}。最后,有效场论不能违背更高能标理论的对称性。原因在于,有效场论和完整理论的红外行为一定要一致。由此,我们可以给出有效场论所满足的几个基本原则:
\begin{enumerate}
	\item 最相关的自由度:有效场论一般只涉及最相关的自由度来保证这个理论对于想要研究的问题足够“有效”;
	\item 对称性允许的所有可能的相互作用;
	\item 数幂原则;
	\item 仅在一定能标范围内有效;
	\item 保留高能标动力学的对称性。
\end{enumerate}

本文的着重点在于有效场论中的一个小类:非相对论性有效场论(nonrelativistic EFT,NREFT)。确切的说,下文中提及的NREFT一般仅包含NRQED和NRQCD\footnote{非相对论性的有效场论还有很多,例如pionless EFT、NRGR等,本文仅会在第三章简要提及前者,而不再涉及其他的非相对论性的EFT。}。非相对论性的理论对我们来说并不陌生,我们最开始接触的量子理论就是非相对论性量子力学(nonrelativistic quantum mechanics)。非相对论性量子力学的有效性在于它所处理的能标远小于电子质量,相对论效应不显著。NREFT同理。通常来说,如果我们在处理一个带重粒子的系统,我们就可以使用NREFT。这里的重是相对于我们想要处理的能标的:比如NRQED处理的是非相对论性的电子,尽管电子质量相对其他基本粒子来说很小(0.5MeV),但是相对于通常NRQED对应的能标($\sim$keV),电子质量是一个大量。对于包含两个及以上重粒子的系统,类似HQET的半经典近似是不可行的,这时候通常非相对论近似就是最好的选择。过去几十年里,人们以NRQED为平台重复了很多量子力学中的结论\cite{Pineda1998},展示了有效场论在处理低能标问题时的优越性。NRQCD则在夸克偶素的研究中起到了重要作用,NRQCD因子化成功地将夸克偶素反应过程中非微扰的强子化部分和微扰的部分子散射分离开来,使关于夸克偶素的计算成为可能。本文即将从NRQED出发,重现一些非相对论性量子力学中的经典结论,并解决相对论性量子力学波函数近原点发散的谜团;同时,本文将把NRQCD因子化从夸克偶素扩展到全重四夸克态,并计算相应的产生过程。

本文内容安排如下:第\ref{chap_NREFT}章将简要地回顾NREFT及其量子化,NRQCD因子化;第\ref{chap_OPE}章利用算符乘积展开(operator product expansion,OPE)和NREFT研究原子波函数的近原点渐进行为;第\ref{chap_Tetraquark}章利用NRQCD研究全重四夸克态在LHC和B工厂上的产生过程。